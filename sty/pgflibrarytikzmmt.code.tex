%%%%%%%%%%%%%%%%%%%%%%%%%%%%%%%%%%%%%%%%%%%%%%%%%%%%%%%%%%%%%%%%%
% A TIKZ library for MMT Theory Graphs
% copyright 2014 Michael Kohlhase; Released under the LPPL
%%%%%%%%%%%%%%%%%%%%%%%%%%%%%%%%%%%%%%%%%%%%%%%%%%%%%%%%%%%%%%%%%
%
% this library provides some standardized node and arrow styles for formatting MMT graph
% diagrams in tikz. The advantage is that we can classify the arrows and nodes
% symbolically and with the styles in this library achieve a uniform look that helps
% readability. 
%
%%%%%%%%%%%%%%%%%%%%%%%%%%%%%%%%%%%%%%%%%%%%%%%%%%%%%%%%%%%%%%%%%

%%% 1. Theories
% a generic theory
\def\outerthysep{.3mm}
\def\innerthysep{.5mm}
\tikzstyle{thy}=[draw,outer sep=\outerthysep,rounded corners,inner sep=\innerthysep]
% a primitive theory
\tikzstyle{primthy}=[thy,double]
% a theory graph 
\tikzstyle{thygraph}=[draw,outer sep=1mm,rounded corners,dashed]

%%% 2. Arrows 
%%% 2.1. Arrowtips (only internal)
\usetikzlibrary{arrows}
\newcommand{\@mmtarrowtip}{angle 45}
\newcommand{\@mmtreversearrowtip}{angle 45 reversed}
\newcommand{\@mmtarrowtipepi}{triangle 45}
\newcommand{\@mmtarrowtipmonoright}{right hook}
\newcommand{\@mmtarrowtipmonoleft}{left hook}
\newcommand{\@mmtarrowtippartial}{right to}
\newcommand{\@mmtarrowtippartialleft}{left to}
\newcommand{\@mmtreversearrowtippartial}{right to reversed}
\newcommand{\@mmtreversearrowtippartialleft}{left to reversed}

%%% 2.2 the arrow sstyles in graphs
\usetikzlibrary{decorations,decorations.pathmorphing,decorations.markings}
% any morphism 
\tikzstyle{morph}=[-\@mmtarrowtip,thick] 
\tikzstyle{mapsto}=[|-\@mmtarrowtip] %| any morphism 
% structures
\tikzstyle{struct}=[-\@mmtarrowtip,thick]
% inclusions: regular, partial, and left variants
\tikzstyle{include}=[\@mmtarrowtipmonoright-\@mmtarrowtip,thick]
\tikzstyle{revinclude}=[\@mmtarrowtip-\@mmtarrowtipmonoright,thick]
\tikzstyle{pinclude}=[\@mmtarrowtipmonoright-\@mmtarrowtippartial,thick]
\tikzstyle{includeleft}=[\@mmtarrowtipmonoleft-\@mmtarrowtip,thick]
\tikzstyle{pincludeleft}=[\@mmtarrowtipmonoleft-\@mmtarrowtippartialleft,thick]
% views: regular, mono, partial, and left variants
\tikzstyle{preview}=[decorate,
                                decoration={coil,aspect=0,amplitude=1pt,
                                                    segment length=6pt,
                                                    pre=lineto,pre length=3pt,
                                                    post=lineto,post length=5pt},
                                thick]
\tikzstyle{view}=[preview,-\@mmtarrowtip]
\tikzstyle{mview}=[preview,\@mmtarrowtipmonoright-\@mmtarrowtip]
\tikzstyle{pview}=[preview,-\@mmtarrowtippartial]
\tikzstyle{pmview}=[preview,\@mmtarrowtipmonoright-\@mmtarrowtippartial]
\tikzstyle{viewleft}=[preview,-\@mmtarrowtip]
\tikzstyle{mviewleft}=[preview,\@mmtarrowtipmonoleft-\@mmtarrowtip]
\tikzstyle{pviewleft}=[preview,-\@mmtarrowtippartialleft]
\tikzstyle{pmviewleft}=[preview,\@mmtarrowtipmonoleft-\@mmtarrowtippartialleft]
\tikzstyle{adoption}=[preview,thin,double,-\@mmtarrowtip]
% biviews: regular, partial, and left variants
\tikzstyle{biview}=[preview,\@mmtarrowtip-\@mmtarrowtip]
\tikzstyle{pbiview}=[preview,\@mmtreversearrowtippartial-\@mmtarrowtippartial]
\tikzstyle{pbiviewleft}=[preview,\@mmtreversearrowtippartialleft-\@mmtarrowtippartialleft]
% defining views (experimental)
\tikzstyle{defview}=[preview,densely dotted,-\@mmtarrowtip]
% meta-theory inclusion
\tikzstyle{meta}=[dotted,-\@mmtarrowtip,thick]
% conservative extensions (abbreviation)
\tikzstyle{conservative}=[hooks-\@mmtarrowtip,double] 
% conservative development
\tikzstyle{conservdev}=[hooks-\@mmtarrowtip,dashed,double] 

% antimorphisms as striktthroughs
\tikzset{anti/.style={
    decoration={markings, mark=between positions 0.2 and 0.8 step 4mm with {
        \draw [thick,-] ++ (-3pt,-3pt) -- (3pt,3pt);}},
    postaction={decorate}}}

% parallel markup
\tikzstyle{parallel}=[\@mmtarrowtip-\@mmtarrowtip,dashed]

%%%% 3. Realms 
\tikzstyle{prerealm}=[draw=blue!40,rectangle,rounded corners,inner sep=10pt,inner ysep=20pt]
\tikzstyle{realm}=[prerealm,fill=gray!4]
\tikzstyle{pillar}=[prerealm,fill=gray!10]

%%% 4. convenience macros
%%% 4.1 the \mmtthy macro takes three arguments, name, decl, axioms and makes a 
% table-like structure
\newcommand\mmtthy[3]{\def\@second{#2}\def\@third{#3}%
\begin{array}{l}\textsf{#1}%
\ifx\@second\@empty\else\\\hline #2\fi%
\ifx\@third\@empty\else\\\hline #3\fi%
\end{array}}
%%% 4.2 the \mmtar takes two arguments, some tikz options, and an arrow style. \nmmtar
% is a variant that also has a name on top.
\newcommand\mmtar[2][]{\raisebox{.5ex}{\tikz[#1]{\draw[#2] (0,0) -- (.6,0);}}}
\newcommand\nmmtar[3][]{\raisebox{.4ex}{\tikz[#1]{\draw[#2] (0,0) --
      node[above]{\ensuremath{\scriptstyle #3}} (.8,0);}}}
%%%% 3.3 Pushout symbols
%%  (after http://tex.stackexchange.com/questions/1144/pushouts-and-pullbacks)
%% \nepushout[<pos>]{<basenode>}{<targetnode>} positions a pushout symbol 
%% (available separately as \pushoutsymb) <pos> of the way between <basenode>
%%  and <targetnode>. 
\usetikzlibrary{calc}
\newcommand\@pushout[4][]{\def\@test{#1}%
\ifx\@test\@empty\def\@@num{.5}\else\def\@@num{#1}\fi%
\begin{scope}[shift=($(#2)!\@@num!(#3)$)]\@nameuse{@#4pushoutsymb}\end{scope}}
\newcommand\@nepushoutsymb{\draw +(-.2,0) -- +(0,0)  -- +(0,-.2);\fill +(-.1,.1) circle (.03);}
\newcommand\nepushoutsymb{\tikz{\@nepushoutsymb}}
\newcommand\nepushout[3][]{\@pushout[#1]{#2}{#3}{ne}}
\newcommand\@sepushoutsymb{\draw +(-.2,0) -- +(0,0)  -- +(0,.2);\fill +(-.1,.1) circle (.03);}
\newcommand\sepushoutsymb{\tikz{\@sepushoutsymb}}
\newcommand\sepushout[3][]{\@pushout[#1]{#2}{#3}{se}}
\newcommand\@npushoutsymb{\draw +(-.1,-.1) -- +(0,0)  -- +(.1,-.1);\fill +(0,-.1) circle (.03);}
\newcommand\npushoutsymb{\tikz{\@sepushoutsymb}}
\newcommand\npushout[3][]{\@pushout[#1]{#2}{#3}{n}}
